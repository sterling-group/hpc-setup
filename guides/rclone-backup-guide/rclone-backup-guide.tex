\documentclass[12pt,letterpaper]{article}
\usepackage[utf8]{inputenc}
\usepackage[T1]{fontenc}
\usepackage{lmodern}
\usepackage{graphicx}
\usepackage{caption}
\captionsetup{
    font=small,
    labelfont=bf,
    textfont=md,
    format=hang,
    justification=centering,
    textformat=period,
}
\usepackage{float}
\usepackage{geometry}
\usepackage{microtype}
\usepackage[breaklinks=true,hidelinks]{hyperref}
\usepackage{listings}
\usepackage{xcolor}
\usepackage[most]{tcolorbox}
\usepackage{amsmath}
\usepackage{cleveref}
\usepackage{textcomp}
\newcommand{\mytexttilde}{\raisebox{0.5ex}{\texttildelow}}
% Set page margins
% \geometry{
%     a4paper,
%     left=25mm,
%     right=25mm,
%     top=25mm,
%     bottom=25mm,
% }

% Define custom style for listings
\lstdefinestyle{custombash}{
    language=bash,
    basicstyle=\ttfamily\small,
    keywordstyle=\color{blue},
    commentstyle=\color{gray},
    stringstyle=\color{red},
    backgroundcolor=\color{gray!10},
    frame=single,
    breaklines=true,
    showstringspaces=false,
}

\title{\LARGE Rclone Backup to Box for Cluster}
\author{Markus G. S. Weiss}
\date{2025/05/05}

\begin{document}

\maketitle

\tableofcontents
\newpage

\section{Introduction}
This tutorial explains how to configure \texttt{rclone} on your cluster to back up
\texttt{/mfs/io/groups/sterling/mfshome/\$USER} to a Box directory named \texttt{cluster-backup},
with subfolders for \texttt{daily}, \texttt{archive}, and \texttt{logs}, and how to
schedule it via \texttt{cron}. Users in the sterling group only need to run the commands in
Sections~\ref{sec:configure}--\ref{sec:prepare} and~\ref{sec:install-cron}. The scripts are
maintained centrally under \verb|/mfs/io/groups/sterling/setup|.

\section{Prerequisites}
\begin{itemize}
  \item \texttt{rclone} (v1.38 or later) installed on both the cluster and your desktop (with a browser).
  \item Confirm \texttt{rclone} versions match:
    \begin{lstlisting}[style=custombash]
/mfs/io/groups/sterling/software-tools/rclone/rclone-v1.69.1-linux-amd64/rclone version
rclone version
    \end{lstlisting}
  \item A Box Enterprise SSO account.
  \item Shell access to the cluster with \texttt{cron} available.
\end{itemize}

\begin{tcolorbox}[title=Tip]
Before running any live syncs, you can test with \texttt{--dry-run} to see what would transfer
or delete without affecting Box:
\begin{lstlisting}[style=custombash]
/mfs/io/groups/sterling/software-tools/rclone/rclone-v1.69.1-linux-amd64/rclone sync \
  /mfs/io/groups/sterling/mfshome/$USER box:cluster-backup/daily \
  --dry-run --fast-list --checksum
\end{lstlisting}
\end{tcolorbox}

\section{Configure the Box remote with offline authorization}
\label{sec:configure}
On the cluster, run \texttt{rclone} using its full path:
\begin{lstlisting}[style=custombash]
/mfs/io/groups/sterling/software-tools/rclone/rclone-v1.69.1-linux-amd64/rclone config
\end{lstlisting}

Press Enter to accept each default (shown in \texttt{<angle brackets>}):
\begin{lstlisting}
No remotes found, make a new one?
n/s/q> n

name> box
Storage> box
client_id> <leave blank>
client_secret> <leave blank>
box_config_file> <leave blank>
access_token> <leave blank>

box_sub_type>
  1 / user
  2 / enterprise
box_sub_type> 2

Edit advanced config?
y/n> n

Use web browser to automatically authenticate?
y/n> n
\end{lstlisting}

\begin{enumerate}
  \item \textbf{Copy} the printed \texttt{rclone authorize} command, switch to your local machine, paste and run it. Complete the OAuth flow in your browser to obtain a long token string.
  \item Back on the cluster, paste only that token at:
    \begin{verbatim}
config_token> xxxxxxxxxxxxxxxx...xxx
    \end{verbatim}
  \item When asked, keep the remote:
    \begin{verbatim}
Keep this "box" remote?
y) y
    \end{verbatim}
  \item Verify:
    \begin{lstlisting}[style=custombash]
/mfs/io/groups/sterling/software-tools/rclone/rclone-v1.69.1-linux-amd64/rclone lsd box:
    \end{lstlisting}
\end{enumerate}

\section{Create the Box folder hierarchy}
\label{sec:create-hierarchy}
Run once on the cluster using the full \texttt{rclone} path:
\begin{lstlisting}[style=custombash]
# Parent backup folder
/mfs/io/groups/sterling/software-tools/rclone/rclone-v1.69.1-linux-amd64/rclone mkdir box:cluster-backup

# Subfolders
/mfs/io/groups/sterling/software-tools/rclone/rclone-v1.69.1-linux-amd64/rclone mkdir box:cluster-backup/daily
/mfs/io/groups/sterling/software-tools/rclone/rclone-v1.69.1-linux-amd64/rclone mkdir box:cluster-backup/archive
/mfs/io/groups/sterling/software-tools/rclone/rclone-v1.69.1-linux-amd64/rclone mkdir box:cluster-backup/logs
\end{lstlisting}
Verify:
\begin{lstlisting}[style=custombash]
/mfs/io/groups/sterling/software-tools/rclone/rclone-v1.69.1-linux-amd64/rclone lsd box:cluster-backup
\end{lstlisting}

\section{Prepare the local environment}
\label{sec:prepare}
On the cluster, create a directory for logs:
\begin{lstlisting}[style=custombash]
mkdir -p ~/logs
\end{lstlisting}

\section{Reference scripts}
\label{sec:reference-scripts}
Sterling group members \emph{do not} need to modify these; they live in
\verb|/mfs/io/groups/sterling/setup|.

\subsection{A) \texttt{backup.sh}}
\begin{lstlisting}[style=custombash]
#!/usr/bin/env bash

# ------------------------------------------------------------------------------
# Script: backup.sh
# Description:
#   Backs up local data to remote storage via rclone.
#   - Daily incremental backups
#   - Weekly snapshots (Sundays)
#   - Prunes local logs older than 30 days
#   - Prunes remote snapshots older than 28 days
#   - Uploads logs to remote
# Usage:
#   backup.sh  (override settings via environment variables as needed)
#
# Configuration (env overrides):
#   DATA_DIR           Local directory to back up (default: /mfs/.../$USER)
#   REMOTE_ROOT        Remote root for backups (default: box:cluster-backup)
#   RCLONE_BIN         Path to rclone binary (default: rclone-v1.69.1)
#   LOG_DIR            Directory for local logs (default: $HOME/logs)
#
# Author: Markus G. S. Weiss
# Date:   2025-05-05
# ------------------------------------------------------------------------------
set -euo pipefail

# --- Configuration (override via env if desired) ------------------------------
: "${DATA_DIR:=/mfs/io/groups/sterling/mfshome/$USER}"
: "${REMOTE_ROOT:=box:cluster-backup}"
: "${RCLONE_BIN:=/mfs/io/groups/sterling/software-tools/rclone/rclone-v1.69.1-linux-amd64/rclone}"
: "${LOG_DIR:=$HOME/logs}"

DATE_STR=$(date +%F)
LOCK_FILE="$HOME/.backup_${USER}.lock"

# Common rclone options
RCLONE_OPTS="--dry-run --fast-list --checksum --log-level WARNING"

# Retry settings
MAX_RETRIES=3
RETRY_DELAY=10

# Snapshot retention (days)
REMOTE_RETENTION_DAYS=28

# --- Setup --------------------------------------------------------------------
# Ensure log directory exists
mkdir -p "$LOG_DIR"

# Prevent overlapping runs
exec 200>"$LOCK_FILE"
flock -n 200 || {
  echo "[$(date '+%F %T')] Another backup is already running. Exiting." >> "$LOG_DIR/backup-$DATE_STR.log"
  exit 1
}

# --- Utility: retry wrapper ---------------------------------------------------
retry() {
  local n=1 cmd="$*"
  until eval "$cmd"; do
    if (( n >= MAX_RETRIES )); then
      echo "[$(date '+%F %T')] ERROR: Command failed after $MAX_RETRIES attempts: $cmd" >> "$LOG_DIR/backup-$DATE_STR.log"
      return 1
    fi
    echo "[$(date '+%F %T')] WARN: Command failed (attempt $n/$MAX_RETRIES). Retrying in $RETRY_DELAY s..." >> "$LOG_DIR/backup-$DATE_STR.log"
    sleep $RETRY_DELAY
    ((n++))
  done
}

# --- 1) Prune local logs older than N days ------------------------------------
prune_local_logs() {
  local retention_days=30 logf="$LOG_DIR/backup-$DATE_STR.log"
  echo "[$(date '+%F %T')] Pruning local logs older than $retention_days days..." >> "$logf"
  find "$LOG_DIR" -type f -name '*.log' -mtime +$retention_days -delete
  echo "[$(date '+%F %T')] Pruning local logs completed." >> "$logf"
}

# --- 2) Prune old remote snapshots --------------------------------------------
prune_remote_snapshots() {
  local logf="$LOG_DIR/backup-$DATE_STR.log"
  echo "[$(date '+%F %T')] Pruning remote snapshots older than $REMOTE_RETENTION_DAYS days..." >> "$logf"
  retry "$RCLONE_BIN delete '$REMOTE_ROOT/archive' --min-age ${REMOTE_RETENTION_DAYS}d $RCLONE_OPTS" >> "$logf"
  echo "[$(date '+%F %T')] Pruned remote snapshots." >> "$logf"
}

# --- 3) Daily incremental backup ----------------------------------------------
backup_daily() {
  local src="$DATA_DIR" dest="$REMOTE_ROOT/daily" logf="$LOG_DIR/backup-$DATE_STR.log"
  echo "[$(date '+%F %T')] Starting daily backup from $src to $dest..." >> "$logf"
  retry "$RCLONE_BIN sync '$src' '$dest' $RCLONE_OPTS --log-file '$logf'"
  echo "[$(date '+%F %T')] Daily backup completed." >> "$logf"
}

# --- 4) Weekly snapshot (Sundays) --------------------------------------------
snapshot_weekly() {
  if [[ "$(date +%u)" == "7" ]]; then
    local src="$DATA_DIR" dest="$REMOTE_ROOT/archive/$DATE_STR" logf="$LOG_DIR/snapshot-$DATE_STR.log"
    echo "[$(date '+%F %T')] Starting weekly snapshot from $src to $dest..." >> "$logf"
    retry "$RCLONE_BIN copy '$src' '$dest' $RCLONE_OPTS --log-file '$logf'"
    echo "[$(date '+%F %T')] Weekly snapshot completed." >> "$logf"
  fi
}

# --- 5) Upload logs ----------------------------------------------------------
upload_logs() {
  local src="$LOG_DIR" dest="$REMOTE_ROOT/logs" logf="$LOG_DIR/backup-$DATE_STR.log"
  echo "[$(date '+%F %T')] Uploading logs from $src to $dest..." >> "$logf"
  retry "$RCLONE_BIN sync '$src' '$dest' $RCLONE_OPTS --log-file '$logf'"
  echo "[$(date '+%F %T')] Log upload completed." >> "$logf"
}

# --- Main --------------------------------------------------------------------
main() {
  prune_local_logs
  prune_remote_snapshots
  backup_daily
  snapshot_weekly
  upload_logs
  echo "[$(date '+%F %T')] Script finished successfully." >> "$LOG_DIR/backup-$DATE_STR.log"
}

main "$@"

# ---Log Rotation (optional) -------------------------------------------------
# For home-directory logs, add ~/.config/logrotate/backup:
# $HOME/logs/*.log {
#   daily
#   rotate 30
#   compress
#   missingok
#   notifempty
#   copytruncate
# }
\end{lstlisting}

Make it executable:
\begin{lstlisting}[style=custombash]
chmod +x /mfs/io/groups/sterling/setup/backup.sh
\end{lstlisting}

\subsection{B) \texttt{cronscript}}
\begin{lstlisting}[style=custombash]
# ------------------------------------------------------------------------------
# Crontab: sterling's backup jobs
# Description:
#   Runs the master backup.sh every day, with all pruning and log-uploads
#   handled internally in that script.
#
# Author: Markus G. S. Weiss
# Date:   2025-05-05
# ------------------------------------------------------------------------------

SHELL=/bin/bash
PATH=/usr/local/bin:/usr/bin:/bin
MAILTO=$USER@utdallas.edu

# Run backup.sh daily at 02:00
0 2 * * * /mfs/io/groups/sterling/setup/backup.sh
\end{lstlisting}

\section{Install the cron job}
\label{sec:install-cron}
On the cluster, install the pre-written cron script:
\begin{lstlisting}[style=custombash]
crontab /mfs/io/groups/sterling/setup/cronscript
\end{lstlisting}
Verify:
\begin{lstlisting}[style=custombash]
crontab -l
\end{lstlisting}

\section{Monitoring \& Maintenance}
\begin{itemize}
  \item \textbf{View logs (live tail):}
    \begin{lstlisting}[style=custombash]
tail -f ~/logs/backup-$(date +%F).log
    \end{lstlisting}
  \item \textbf{Clean up local logs older than 30 days:}
    \begin{lstlisting}[style=custombash]
find ~/logs -type f -mtime +30 -delete
    \end{lstlisting}
  \item \textbf{Test restores:}
    \begin{lstlisting}[style=custombash]
/mfs/io/groups/sterling/software-tools/rclone/rclone-v1.69.1-linux-amd64/rclone copy \
  box:cluster-backup/daily/path/to/file /tmp && \
  diff /tmp/file /mfs/io/groups/sterling/mfshome/$USER/path/to/file
    \end{lstlisting}
  \item \textbf{Error notifications:} Cron will email stderr/stdout to \verb|$USER@yourdomain.com|. For advanced alerting, grep logs for \texttt{ERROR} and pipe to mail or integrate with Slack.
\end{itemize}

\section{Additional Notes}
\begin{itemize}
  \item \textbf{Security \& permissions:}
    Do \emph{not} check \verb|~/.config/rclone/rclone.conf| into any shared repositories—it contains tokens.
  \item \textbf{Data encryption:}
    Consider using an rclone \texttt{crypt} wrapper for encryption at rest.
  \item \textbf{API rate limits:}
    Box enforces API quotas. Tweak \texttt{--transfers}, \texttt{--checkers}, or add \texttt{--tpslimit 3} if you hit rate-limit errors.
  \item \textbf{Network/firewall:}
    Ensure outbound HTTPS (port~443) is open. If behind a proxy, set \verb|HTTPS_PROXY| or use \texttt{--proxy}.
  \item \textbf{Monthly or quarterly snapshots:}
    Extend the weekly logic with checks like:
    \begin{lstlisting}[style=custombash]
if [[ "$(date +%d)" == "01" ]]; then
  ... # monthly archive
fi
    \end{lstlisting}
  \item \textbf{Upstream docs:}
    Official rclone Box backend documentation:
    \href{https://rclone.org/box/}{https://rclone.org/box/}
\end{itemize}

\section{Summary}
In this tutorial, you have:
\begin{itemize}
  \item \textbf{Configured} the Box remote on a headless cluster node via offline authorization.
  \item \textbf{Created} a clear Box folder hierarchy (\texttt{cluster-backup/\{daily,archive,logs\}}) for organized storage.
  \item \textbf{Prepared} a local log directory and referenced centrally maintained backup and cron scripts.
  \item \textbf{Written} a robust \texttt{backup.sh} that performs daily incremental syncs, weekly snapshots, and log uploads.
  \item \textbf{Scheduled} the backup using a \texttt{crontab}, including log rotation and snapshot cleanup.
  \item \textbf{Implemented} monitoring, restore procedures, and maintenance routines (log pruning, error alerts).
  \item \textbf{Added} best-practice notes on dry-runs, version checks, security, API-rate limits, and firewall considerations.
\end{itemize}

Great work! Your cluster’s home directory is now automatically and safely backed up to Box every night, with versioning, logs, and the tools for easy maintenance and recovery.

\end{document}
